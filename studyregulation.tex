\chapter{Study regulation}
The student must achieve knowledge on following theories and methods:

\noindent Object-oriented modelling in analysis and design:
\begin{itemize}
    \item Modelling of context (problem domain and application domain) 
    \item Object-oriented concepts: class, object, events, structures, function, use cases, components and component architecture 
    \item UML: class diagram, state chart diagrams, sequence diagrams, use case diagrams 
\end{itemize}

\noindent Modelling with patterns:

\begin{itemize}
    \item Patterns for modelling the problem domain and the application domain 
    \item Patterns for connection of components 
    \item Including analysis patterns: item-descriptor, hierarchy, stepwise role, material, procedure. 
    \item Including design patterns: composite, layered architecture, observer, client-server, model-view-controller 
\end{itemize}

\noindent System development method:

\begin{itemize}
    \item Waterfall and model-driven development
    \item Iterative method and prototype-driven method
\end{itemize}

\noindent The student must achieve the following skills:
\begin{itemize}
    \item Be able to explain himself or herself precisely and use relevant concepts and modelling language
    \item Be able to model requirements for a system, its context and its different parts (model, functions and interfaces) 
    \item Be able to model a system design at component level and describe connections between components. 
\end{itemize}

\noindent The student must be able to use the concepts, the patterns, and the modelling language to describe a concrete system that solves a well-defined task.